
\documentclass{emulateapj}



\slugcomment{SPD 2017 Abstract}


\shorttitle{Collapsed Cores in Globular Clusters}
\shortauthors{Djorgovski et al.}



\begin{document}
	

	\title{Solar Physics Division Meeting 2017 \\ Measuring Plasma Flows in Transition Region Loops Using the MOSES Instrument}
	

	\author{Roy Smart}
	\affil{Physics Department, Montana State University, Bozeman, MT 59717}
	\email{roy.smart@montana.edu}
	
	\author{Charles Kankelborg}
	\affil{Physics Department, Montana State University, Bozeman, MT 59717}
	
	\and
	
	\author{Nicholas Bonham}
	\affil{Physics Department, Montana State University, Bozeman, MT 59717}
	

	
	\begin{abstract}
		Coronal loops are an important feature of the solar corona. Accurately modeling these features would require measurements of plasma parameters such as density, velocity and temperature over the entire spatial extent of a loop. While traditional slit-spectrographs have been extremely valuable for observing coronal loops, the narrow slit of these instruments prevents the properties of coronal loops to be measured across their entire structure.
		
		To gain a more comprehensive understanding of the structure of coronal loops, we will utilize the Multi-Order Solar EUV Spectrograph (MOSES), which is a sounding rocket-based solar hyperspectral imager. Since hyperspectral images provide spectral information over a wide field-of-view, observations using MOSES allow for the determination of plasma velocity over the entire structure of a coronal loop. In this work we present our technique for extracting doppler velocities from MOSES observations, measurements of downflow velocities for two loop footpoints captured during the MOSES II flight in 2015, and comparisons of those measurements to a 1D radiative hydrodynamic loop model.
		

	\end{abstract}
	
	\section{}
	
\end{document}
